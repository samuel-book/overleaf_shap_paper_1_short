`\section{Introduction}

% Include
% 1) What is the problem?
% 2) What do we know about low and varying use of thrombolysis
% 3) What do we not know
% 4) How are we addressing what we don't know

% 1) What is the problem?

Stroke remains one of the top three global causes of death and disability \cite{feigin_global_2021}. Despite reductions in age-standardised rates of stroke, ageing populations are driving an increase in the absolute number of strokes \cite{feigin_global_2021}. Across Europe, in 2017, stroke was found to cost healthcare systems \texteuro 27 billion, or 1.7\% of health expenditure \cite{luengo-fernandez_economic_2020}. Thrombolysis with recombinant tissue plasminogen activator, can significantly reduce disability after ischaemic stroke, so long as it is given in the first few hours after stroke onset \cite{emberson_effect_2014}. Despite thrombolysis being of proven benefit in ischaemic stroke, use of thrombolysis varies significantly both between and within European countries \cite{aguiar_de_sousa_access_2019}. In England and Wales the national stroke audit reported that in 2021/22, 20 years on from the original European Medicines Agency licencing of alteplase for acute ischaemic stroke, thrombolysis rates for emergency stroke admissions varied from just 1\% to 28\% between hospitals, \cite{sentinel_national_stroke_audit_programme_ssnap_2022}, with a median rate of 10.4\% and an inter-quartile range of 8\%-13\%, against a 2019 NHS England long term plan that 20\% of patients of emergency stroke admissions should be receiving thrombolysis \cite{nhs_long_term_plan_2019}.

% 2) What do we know about low and varying use of thrombolysis

Studies have shown that reasons for low and varying thrombolysis rates are multi-factorial. Reasons include late presentation \cite{aguiar_de_sousa_access_2019}, lack of expertise \cite{aguiar_de_sousa_access_2019} or lack of clear protocols or training \cite{carter-jones_stroke_2011}, delayed access to specialists \cite{kamal_delays_2017}, and poor triage by ambulance or emergency department staff \cite{carter-jones_stroke_2011}. For many factors, the establishment of primary stroke centres has been suggested to improve the emergency care of patients with stroke and reduce barriers to thrombolysis \cite{carter-jones_stroke_2011}, with a centralised model of primary stroke centres leading to increased likelihood of thrombolysis \cite{lahr_proportion_2012, morris_impact_2014, hunter_impact_2013}. 

In addition to organisational factors, clinicians can have varying attitudes to which patients are suitable candidates for thrombolysis. In a discrete choice experiment \cite{de_brun_factors_2018}, 138 clinicians considered hypothetical patient vignettes, and responded as to whether they would give the patients thrombolysis. The authors concluded that there was considerable heterogeneity among respondents in their thrombolysis decision-making. Areas of difference were around whether to give thrombolysis to mild strokes, to older patients beyond 3 hours from stroke onset, and when there was pre-existing disability.

Based on national audit data from three years of emergency stroke admissions, we have previously built models of the emergency stroke pathway using clinical pathway simulation to examine the potential scale of the effect of changing two aspects of the stroke pathway performance (1. the in-hospital process speeds, and 2. the proportion of patients with a known stroke onset time), and using machine learning to examine the effect of replicating clinical decision-making around thrombolysis from higher thrombolysing hospitals to lower thrombolysing hospitals \cite{allen_using_2022, allen_use_2022}. The machine learning model learns whether any particular patient would receive thrombolysis in any emergency stroke centre. Using these models we found that it would be credible to target an increase in average thrombolysis in England and Wales, from 11\% to 18\%, but that each hospital should have its own target, reflecting differences in local populations. We found that the largest increase in thrombolysis use would come from replicating thrombolysis decision-making practice from higher to lower thrombolysing hospitals. Two other important factors influencing thrombolysis rates were determination of stroke onset time in some hospitals, and improving the speed of the in-hospital thrombolysis pathway.

% 3) What do we not know

In our previous work we established that we could predict the use of thrombolysis in patients arriving within 4 hours of known stroke onset with 85\% accuracy. We could then ask the question "What if this patient attended another hospital - would they likely be given thrombolysis?" As this was a \emph{`black-box'} decision-forest model we could not effectively explain the relationship between patient characteristics and their chance of receiving thrombolysis, or identify and explain the patient characteristics which different hospitals would differ on.

% 4) How are we addressing what we don't know

In this paper, therefore, we seek to use \emph{explainable machine learning} to understand the relationship between patient characteristics (`features') and the use of thrombolysis across England and Wales, and we seek to understand how hospitals differ in their attitudes to use of thrombolysis, and how much difference in use of thrombolysis may be explained by those differences. We use an \emph{eXtreme Gradient Boosting model \cite{chen_xgboost_2016}} (XGBoost) to make predictions and then use an additional \emph{SHapley Additive exPlanations} \cite{lundberg_unified_2017} (SHAP) model to explain the contribution of each patient characteristic to the model prediction.