\section{Results}

\subsection{Variation in observed hospital thrombolysis use}

Thrombolysis use in the original data varied between hospitals from 1.5\% to 24.3\% of all patients, and 7.3\% to 49.7\% of patients arriving within 4 hours of known (precise or estimated) stroke onset.

%%%%%%%%%%%%%%%%%%%%%%%%%%%%%%%%%%%%%%%%%%%%%%%%%%%%%%%%%%%%%%%%%%%%%%%%

\subsection{Feature selection}

The best model with 1, 2, 5, 10, 25 \& all 60 original features had ROC AUCs of 0.715, 0.792, 0.891, 0.919, 0.923 \& 0.922. We selected 10 features for all subsequent work, which were:

\begin{itemize}
    \item \emph{Arrival-to-scan time}: Time from arrival at hospital to scan (minutes)
    \item \emph{Infarction}: Stroke type (1 = infarction, 0 = haemorrhage)
    \item \emph{Stroke severity}: National Institutes of Health Stroke Scale (NIHSS) score on arrival
    \item \emph{Precise onset time}: Onset time (1 = precise, 0 = best estimate)
    \item \emph{Prior disability level}: Disability level (modified Rankin Scale; mRS) before stroke
    \item \emph{Stroke team}: Stroke team attended (hospital identifier)
    \item \emph{Use of anticoagulants}: Use of prior anticoagulant (1 = Yes, 0 = No)
    \item \emph{Onset-to-arrival time}: Time from onset of stroke to arrival at hospital (mins)
    \item \emph{Onset during sleep}: Did stroke occur in sleep? (1 = Yes, 0 = No)
    \item \emph{Age}: Age (as midpoint of 5 year age bands)
\end{itemize}

Correlations between the 10 features were measured using coefficients of determination (r-squared). All r-squared were less than 0.05 except a) age and prior disability level (r-squared 0.146), and b) onset during sleep and precise onset time (r-squared 0.078).

%%%%%%%%%%%%%%%%%%%%%%%%%%%%%%%%%%%%%%%%%%%%%%%%%%%%%%%%%%%%%%%%%%%%%%%%

\subsubsection{Model accuracy}

Model accuracy was measured using stratified 5-fold cross validation. Overall accuracy was 85.0\% (83.9\% sensitivity and specificity could be achieved simultaneously). The ROC AUC was 0.918. The model predicted hospital thrombolysis use at each hospital with very good accuracy (r-squared = 0.977, figure \ref{fig:thrombolysis_pred_observed}). This maintains the high accuracy of our previously published model with overall accuracy of 84.3\% (83.2\% sensitivity and specificity could be achieved simultaneously), and mean ROC AUC of 0.906 \cite{allen_use_2022}.

\begin{figure}
\centering
\includegraphics[width=0.6\textwidth]{./images/02_xgb_10_features_observed_predicted_rates}
\caption{Comparison of predicted and observed thrombolysis use for 132 hospitals.}
\label{fig:thrombolysis_pred_observed}
\end{figure}

The appendix contains further model accuracy analysis.

%%%%%%%%%%%%%%%%%%%%%%%%%%%%%%%%%%%%%%%%%%%%%%%%%%%%%%%%%%%%%%%%%%%%%%%%
\subsection{Individual patient SHAP values}
SHAP values are presented as how they affect log odds of receiving thrombolysis, but for individual predictions, probability values are more intuitive. Figure \ref{fig:results_waterfall} shows waterfall plots for an example of a patient with low (top) and high (bottom) probability of receiving thrombolysis. Waterfall plots show the influence of features for an individual prediction (in our case, patient). The SHAP model starts with a base prediction of a 24\% probability of receiving thrombolysis, before feature values are taken into account. For the patient with a low probability of receiving thrombolysis, the two most influential features reducing the probability of receiving thrombolysis were a long arrival-to-scan time (138 minutes) and a low stroke severity (NIHSS=2). For the patient with a high probability of receiving thrombolysis, the two most influential features increasing the probability of receiving thrombolysis were a short arrival-to-scan time (17 minutes) and a moderate stroke severity (NIHSS=14). 

\begin{figure}%[!ht]
    \caption*{\footnotesize{\textsf{Patient with low probability of receiving thrombolysis}}}
    \vspace{-4mm} % reduce space after caption
    \includegraphics[width=0.75\textwidth]{./images/03_xgb_10_features_waterfall_probability_low}
    \vspace{2mm}
    \caption*{\footnotesize{\textsf{Patient with high probability of receiving thrombolysis}}}
    \vspace{-4mm} % reduce space after caption
    \includegraphics[width=0.75\textwidth]{./images/03_xgb_10_features_waterfall_probability_high}
\caption{Waterfall plots showing the influence of each feature on the predicted probability of a single patient receiving thrombolysis. Top: An example of a patient with a low probability (2.6\%) of receiving thrombolysis. Bottom: An example of a patient with a high probability (95.7\%) of receiving thrombolysis.}
\label{fig:results_waterfall}
\end{figure}

%\newpage
%%%%%%%%%%%%%%%%%%%%%%%%%%%%%%%%%%%%%%%%%%%%%%%%%%%%%%%%%%%%%%%%%%%%%%%%

\subsection{The relationship between feature values and the odds of receiving thrombolysis}

\begin{minipage}{1\textwidth}
Figure \ref{fig:shap_feature_subfigure} shows the relationship between patient level feature values and their SHAP values. Key observations are (with SHAP influence converted from log-odds to odds):

\begin{itemize}
    \item \emph{Stroke type}: The SHAP values for stroke type show that the model effectively eliminated any probability of receiving thrombolysis for non-ischaemic (haemorrhagic) stroke, with the odds of receiving thrombolysis falling by over 6,000 fold.
    \item \emph{Arrival-to-scan time}: The odds of receiving thrombolysis reduced by about 9 fold over the first 120 minutes of arrival-to-scan time.
    \item \emph{Stroke severity (NIHSS)}: The odds of receiving thrombolysis were lowest at NIHSS 0, increased and peaked at NIHSS 15-25, and then fell again with higher stroke severity (NIHSS above 25). The difference between minimum odds (at NIHSS 0) and maximum odds (at 15-25) of receiving thrombolysis was 25-30 fold.
    \item \emph{Stroke onset time type (precise vs. estimated)}: The odds of receiving thrombolysis were about 3 fold greater for precise onset time than estimated onset time.
    \item \emph{Disability level (mRS) before stroke}: The odds of receiving thrombolysis fell about 6 fold between mRS 0 and 5.
    \item \emph{Use of AF anticoagulants}: The odds of receiving thrombolysis were about 5 fold greater for no use.
    \item \emph{Onset-to-arrival time}: The odds of receiving thrombolysis were similar below 120 minutes, then fell about 3 fold between 120 and 240 minutes.
    \item \emph{Age}: The odds of receiving thrombolysis were similar below 80 years old, then fell about 2 fold between 80 and 110 years old.    
    \item \emph{Onset during sleep}: The odds of receiving thrombolysis were about 4 fold lower for onset during sleep.
    \item \emph{Hospital attended}: There was a 13 fold difference in odds of receiving thrombolysis between hospitals.
\end{itemize}
\end{minipage}

In summary the most thrombolysable patients have: an infarction stroke type; shorter arrival-to-scan times; moderate to severe stroke severity with NIHSS 10-25; a precise onset time; lower pre-stroke disability; were not taking anticoagulant medication; shorter onset-to-arrival times; were younger; did not have onset during sleep; attend a hospital with a high predisposition to use thrombolysis.


% SHAP violon plots

\begin{figure}%[!h]
\centering
\begin{subfigure}{.9\textwidth}
  \centering
    \caption*{\footnotesize{\textsf{Violin plots for relationship between SHAP values and feature values}}}
    \includegraphics[width=1.\textwidth]{./images/03a_xgb_10_features_thrombolysis_shap_violin_all_features}
    
  \label{fig:shap_feature_subfigure_a}
\end{subfigure}
\begin{subfigure}{.9\textwidth}
  \centering
  \caption*{\footnotesize{\textsf{Histogram of hospital SHAP values}}}
    \includegraphics[width=1.\textwidth]{./images/03a_xgb_10_features_hosp_shap_hist}
  \label{fig:shap_feature_subfigure_b}
\end{subfigure}
\caption{Plots showing the relationship between SHAP values and feature values. Top: Violin plots showing the relationship between SHAP values and feature values. The horizontal line shows the median SHAP value. The plots are ordered in ranked feature importance (using the mean absolute SHAP value across all instances). Bottom: Histogram showing the frequency of SHAP values for the one-hot encoded hospital feature.}
\label{fig:shap_feature_subfigure}
\end{figure}

%%%%%%%%%%%%%%%%%%%%%%%%%%%%%%%%%%%%%%%%%%%%%%%%%%%%%%%%%%%%%%%%%%%%%%%%

\subsection{How much of the between-hospital thrombolysis use can be explained by the identity of the hospital attended?}

The mean hospital attended SHAP main effect value correlated with the observed hospital thrombolysis rate with an r-squared of 0.558 (figure \ref{fig:shap_correlation}, left), suggesting that 56\% (P$<$0.0001) of the between-hospital variance in thrombolysis use may be explained by the attended hospitals' SHAP main effect values, i.e. the hospitals' predisposition and/or preparedness to use thrombolysis.

Using the \emph{10k holdout model}, the predicted use of thrombolysis across the 132 hospitals for the identical 10k cohort of patients (not used in training the model) ranged from 10\% to 45\%. The mean hospital attended SHAP main effect value for the 10k cohort of patients correlated very closely with the predicted thrombolysis use in the 10k cohort of patients at each hospital, with an r-squared of 0.971 (figure \ref{fig:shap_correlation}, right), confirming that the hospital attended SHAP main effect value is providing a direct insight into a hospitals' propensity to use thrombolysis.

\begin{figure}[!h]
\centering
\begin{subfigure}{.49\textwidth}
  \centering
    \caption*{\footnotesize{\textsf{Observed thrombolysis}}}
    \includegraphics[width=0.9\textwidth]{./images/03c_xgb_10_features_attended_hosp_shap_maineffect_vs_ivt_rate}
\end{subfigure}
\begin{subfigure}{.49\textwidth}
  \centering
    \caption*{\footnotesize{\textsf{Predicted 10K thrombolysis}}}
    \includegraphics[width=0.9\textwidth]{./images/04a_xgb_10_features_10k_cohort_attended_hosp_shap_maineffect_vs_ivt_rate}
\end{subfigure}

\caption{Correlations between hospital attended SHAP main effect value and the observed thrombolysis use at each hospital. Left: Observed thrombolysis (using the all data model). Right: Predicted 10k cohort thrombolysis rate (using the 10k holdout model).}
\label{fig:shap_correlation}
\end{figure}
%%%%%%%%%%%%%%%%%%%%%%%%%%%%%%%%%%%%%%%%%%%%%%%%%%%%%%%%%%%%%%%%%%%


%\newpage
\subsection{How much of the between-hospital thrombolysis use can be explained by the differences in the hospital identity and processes, and the differences in the patient populations?}

We predicted thrombolysis use using mean subset SHAP values for patients attending each hospital. Figure \ref{fig:shap_multiple_regression} shows that 36\% (P$<$0.0001) of the variance in observed between-hospital thrombolysis use can be explained by the patient population, 74\% (P$<$0.0001) can be explained by hospital identity and processes, and that 95\% (P$<$0.0001) can be explained by the combined information from both the patient population and hospital identity and processes. 

\begin{figure}[!h]
    \centering
    \includegraphics[width=0.9\textwidth]{./images/03f_xgb_10_features_multiple_regression_patient_hospital}
    \caption{Multiple regression of subset SHAP values (mean of patients attending hospital) with hospital observed thrombolysis rate. Left: Subset SHAP values for the eight patient descriptive features (age, stroke severity, prior disability, onset-to-arrival time, stroke type, type of onset time, anticoagulants, and onset during sleep). Middle: Subset SHAP values for the two hospital descriptive features (arrival-to-scan time, and hospital attended). Right: Subset SHAP values for all 10 features (for both hospital and patient descriptive features).}
  \label{fig:shap_multiple_regression}
\end{figure}

%%%%%%%%%%%%%%%%%%%%%%%%%%%%%%%%%%%%%%%%%%%%%%%%%%%%%%%%%%%%%%%%%%%%%%%%
%\newpage
\subsection{Variation in hospital thrombolysis use for patient subgroups}

Figure \ref{fig:results_boxplot} shows boxplots of observed and predicted use of thrombolysis, broken down by patient subgroup. The nine subgroups of patients with one defined non-ideal feature (haemorrhagic stroke, arrival-to-scan time 60-90 mins, mild stroke severity, estimated stroke onset time, some pre-stroke disability, use of anticoagulants, onset-to-arrival time 150-180 mins, over 80 years old, and onset during sleep) all had reduced thrombolysis use than the complete patient population, and combining these non-ideal features reduced thrombolysis use further (illustrated by the patient subgroup with a mild stroke severity and an estimated stroke onset time). There was, however, significant variation between hospitals in use of thrombolysis in each of these subgroups. The observed and predicted thrombolysis use show the same general results, but some small differences existed: 1) The use of thrombolysis in \emph{ideal} patients is a little lower in the observed vs predicted results (mean hospital thrombolysis use = 89\% vs 99\%), 2) The predicted results show a stronger effect of combining non-ideal features, 3) The observed thrombolysis rate shows higher between-hospital variation than the predicted thrombolysis rate. This may be partly explained by the observed thrombolysis rate being based on different patients at each hospital, but may also be partly explained by actual use of thrombolysis being slightly more variable than predicted thrombolysis use (which will follow general hospital patterns, and will not include, for example, between-clinician variation at each hospital).

\begin{figure}[!h]
\centering
\includegraphics[width=0.75\textwidth]{./images/04c_xgb_10_features_10k_cohort_actual_vs_modelled_subgroup_boxplot}
\caption{Boxplot for either observed (top) or predicted (bottom) use of thrombolysis for subgroups of patients. The `ideal patients' subgroup has a mid-level stroke severity (NIHSS in range 10-25), short arrival-to-scan time (less than 30 minutes), stroke caused by infarction, precise stroke onset time known, no pre-stroke disability (mRS 0), not taking any atrial fibrillation anticoagulants, short onset-to-arrival time (less than 90 minutes), younger than 80 years old, and onset not during sleep. The single features are ordered in ranked feature importance (using the mean absolute SHAP value across all instances). }
\label{fig:results_boxplot}
\end{figure}